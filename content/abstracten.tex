\keywordsen{Gene regulatory networks; single-cell RNA-seq data; mutual information; dynamic network construction; cellular heterogeneity}
\categoryen{TP391}
\begin{abstracten}

% Gene regulatory networks (GRNs) are formed by the interactions between genes, 
% which reveal their complex regulatory relationships. 
% The construction of GRNs helps us to understand the mechanism of gene regulation and the structure and function of biological network systems, 
% and also helps us to understand the mechanism of tumor and other complex diseases at the molecular level and guide the screening and development of cancer drugs. 
% The construction of GRNs is therefore a fundamental and core problem in systems biology. 
% In this paper, effective methods are proposed for the construction of GRNs on microarray and single-cell RNA-seq datasets, respectively, 
% to address the difficulties including low accuracy and sparse networks in the study of GRN inference.
% The main work in this paper includes:

Gene regulatory networks (GRNs) are formed by the interactions between genes, 
which reveal their complex regulatory relationships. 
The construction of GRNs helps us to understand the mechanism of gene regulation, the structural and functional information of biological network systems, and also the mechanisms of complex diseases such as tumorigenesis at the molecular level.
The construction of gene regulatory network is one of the core and most important problems in systems biology.
In this paper, effective methods for gene regulatory network construction are proposed on datasets generated by ~DNA~microarray and single-cell ~RNA-seq~ sequencing technologies, 
to address  the difficulties in the study of gene regulatory network inference, such as low accuracy and sparse network.
The innovations and main work in this paper are as follows:

% (1) A novel mutual information-based network structure inference method Loc-PCA-CMI is proposed.
% Local overlapping gene clusters are first identified, and then the local network structure of each cluster is inferred based on the path consistency algorithm of conditional mutual information (PCA-CMI),
% The final GRN is constructed by aggregating the local network structure, i.e., the network of dependencies between genes.
% Loc-PCA-CMI reduces the redundant dependencies in the inference of network structure.
% We evaluate LOC-PCA-CMI on the DREAM3 knockout dataset,
% The performance of Loc-PCA-CMI is compared with other information theory-based network structure inference methods, including ARACNE, MRNET, PCA-CMI and PCA-PMI.
% The experimental results show that Loc-PCA-CMI outperforms the other four benchmark methods on the ~DREAM3~ dataset, especially on networks with gene numbers of 50 and 100.

(1) To address the problem of low accuracy of gene network construction based on mutual information, Loc-PCA-CMI is proposed as a gene regulatory network construction method based on mutual information and local structure.
The method identifies locally overlapping gene clusters based on their co-expression relationships, and then infer the local network structure of each gene cluster based on the path consistency algorithm of conditional mutual information (PCA-CMI), 
and finally constructs the final gene regulatory network by aggregating the local network structure, i.e., the dependency network between genes.
We evaluate LOC-PCA-CMI on the DREAM3 knockout dataset and compare its performance with four other information theory-based methods for inferring network structure.
The experimental results show that Loc-PCA-CMI reduces redundant dependencies in construction on the DREAM3 dataset,
In particular, networks with gene numbers 50 and 100 outperform the other four benchmark methods in the evaluation of AUPR.


% (2) A novel data-driven gene regulatory network construction method D3GRN is proposed, in which the regulatory relationship of each target gene is translated into a functional decomposition problem, 
% and the sub-problems are solved by using the algorithm ARNI that reveals the network interactions. 
% To compensate for the limitations of ARNI in constructing networks only at the unit level, 
% we adopt sampling (bootstrapping) and area-based scoring methods to infer the final networks. 
% Studies on data-driven dynamic network construction provide us with new perspectives on solving regression problems. 
% The experimental results show that D3GRN is competitive with state-of-the-art algorithms on the ~DREAM4~ and ~DREAM5~ benchmark~ datasets in terms of AUPR.

(2) To address the shortcomings of current data-driven methods that cannot construct global networks, 
D3GRN is proposed as a data-driven gene regulatory network construction method using a bootstrapping strategy in this paper.
To compensate for the inability of data-driven methods to construct global networks, 
we adopt an area-based scoring method to infer the final networks. 
The experimental results show that on the DREAM4 and DREAM5 benchmark datasets,

% (3) A novel rare cell identification method DoRC based on single-cell RNA-seq data is proposed.
% A challenging problem is how to identify rare cells and their types from the ultra-large scale scRNA-seq data. 
% In addition, single-cell data population, clustering, and rare cell identification are core tasks in upstream analysis of single-cell data and are necessary steps in constructing gene regulatory networks associated with cell types. 
% Existing algorithms for finding rare cells are time-consuming or memory-intensive.
% The rarity scores generated by DoRC can help biologists focus on downstream analysis, 
% analyzing only partially expressed cell scRNA-seq data on a supermassive scale. 
% In order to distinguish cell types in subsequent downstream analysis, we propose a novel and effective cell clustering method RafClust, 
% which also demonstrates the efficacy of DoRC in delineating human blood dendritic cell subtypes using a single cell expression profile of 68k human blood cells. 

(3) In order to solve the problem of inaccurate cell clustering on single-cell RNA-seq datasets, 
RafClust is proposed as a random forest similarity learning-based single-cell clustering method, 
which uses multiple correlation measures to characterize cells, 
and then uses a random forest regression model to further learn the cell-to-cell similarity matrix based on similarity. 
The matrix is applied with hierarchical clustering to determine the final category of cells. 
The experimental results show that RafClust outperforms the other six benchmark methods in ARI on ten single-cell datasets.

% (4) A novel gene expression regulatory network approach scGRNHunter based on single-cell RNA-seq data is proposed. 
% Identifying cell type characteristics and cellular gene expression activity programs (e.g., life cycle processes, responses to environmental factors) is essential for understanding the composition of cells and tissues,
% and also helpful for understanding the mechanism of gene regulation. 
% While single cells RNA-Seq can be quantified as transcripts in individual cells, 
% the expression profile of each cell may be a mixture of these two types of programs, 
% making them difficult to separate. 
% Here, we propose an algorithm WSSMFA using matrix decomposition to solve this problem. 
% Experiments on publicly available human brain organoids scRNA-Seq datasets show that our proposed scGRNHunter method can accurately infer the identity and activity of subprograms, 
% and build a gene regulatory network based on cell class identity and a gene regulatory network based on cell activity.

(4) To address the problem of time-consuming and memory-consuming algorithms for identifying rare cells from single-cell RNA-seq data, 
DoRC is proposed as an isolated forest-based single-cell rare cell identification method, 
which uses isolated forest to efficiently generate a rarity fraction for each cell and combines it with a threshold method to binary label the cells for sparsity. 
The experimental results show that DoRC is highly efficient in classifying human blood dendritic cell subtypes on a single cell expression profile of 68k human blood cells from single cell RNA-seq data. 
In addition, DoRC can identify rare cells in simulated datasets and is sensitive to cell type characteristics.

(5) To address the problem that gene regulatory networks related to cell type and cell activity cannot be inferred simultaneously from single-cell RNA-seq data, 
a matrix decomposition-based gene regulatory network construction method, scGRNHunter, is proposed. 
The cell type program and the cell activity program are also isolated, and then a gene regulatory network based on each program is constructed in conjunction with the public database TRRUST. 
The experimental results show that our proposed scGRNHunter method can effectively infer the identity and activity of subprograms on the publicly available brain-like organ scRNA-Seq dataset, 
and construct cell type-based and cell activity-based gene regulatory networks based on them.


\end{abstracten}