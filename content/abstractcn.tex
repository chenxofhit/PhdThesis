% 设置中文摘要
\keywordscn{基因调控网络; 单细胞~RNA-seq~数据; 互信息; 动态网络构建; 细胞异质性;}
\categorycn{TP391}
\begin{abstractcn}
基因调控网络~(GRNs)~是基因之间的相互作用所形成的网络,
揭示了基因中复杂的调控关系。
构建基因调控网络有助于我们了解基因调控机制,
了解生物网络系统所蕴涵的结构和功能信息,
从分子水平上理解肿瘤等复杂疾病发生的机理并指导肿瘤药物的筛选和研制。
基因调控网络的构建因此是系统生物学中的一个基础和核心的问题。
本文针对基因调控网络推断研究中准确度低,网络稀疏等难点, 
在微阵列和单细胞~RNA-seq~数据集上, 分别提出了有效的基因调控网络构建方法。
本文完成的主要工作有:

(1)~提出了一种新的基于互信息的网络结构推断方法~Loc-PCA-CMI。
首先识别局部重叠基因簇,然后基于条件互信息~(PCA-CMI)~的路径一致性算法推断每个簇的局部网络结构,
最终通过聚合局部网络结构,也就是基因之间的依赖性网络,来构造最终的基因调控网络。
Loc-PCA-CMI~降低了在网络结构推断中的冗余的依赖关系。
我们在~DREAM3~敲除数据集上对~LOC-PCA-CMI~进行了评估,
将其性能与其它基于信息理论的网络结构推断方法,包括~ARACNE、MRNET、PCA-CMI~和~PCA-PMI,进行了比较。
实验结果证明,~Loc-PCA-CMI~在~DREAM3~数据集上特别是在基因数目为~50~和~100~的网络上表现优于其它四种基准方法。

(2)~提出了一种新的数据驱动的基因调控网络构建方法~D3GRN。
在这个方法里,将每个目标基因的调控关系转化为函数分解问题,
并利用揭示网络相互作用的算法~ARNI~解决各个子问题。
为了弥补~ARNI~仅从单元级构建网络的局限性,
我们采用抽样~(bootstrapping)~和基于面积的评分方法来推断最终的网络。
关于数据驱动的动态网络构建的研究为我们提供了解决回归问题的新视角。
实验结果表明, 在~DREAM4~和~DREAM5~基准~数据集上, D3GRN~在~AUPR~这个评价指标上与最先进的算法相比具有竞争力。

(3)~提出了一种新的基于单细胞~RNA-seq~数据的稀有细胞识别方法~DoRC。
一个具有挑战性的问题是如何从超大规模的~scRNA-seq~数据中识别稀有细胞及其类型。
另外,单细胞数据填充、聚类以及稀有细胞识别是单细胞数据上游分析的核心任务, 
是构建与细胞类型相关的基因调控网络的必要步骤。
现有的寻找稀有细胞的算法非常耗时或耗费内存。
DoRC~产生的稀有度分数可以帮助生物学家们着重于下游分析,只对超大规模内的部分表达细胞~scRNA-seq~数据进行分析。
为了在随后的下游分析的过程中区分细胞类型,我们提出了新颖有效的细胞聚类方法~RafClust。
我们使用~${\sim}68$k~人血细胞的单细胞表达谱还表明了~DoRC~在划分人类血液树突状细胞亚型方面的效果。
DoRC可以识别仿真数据集里面的稀有细胞,并且对细胞类型特征也很敏感。

(4)~提出了一种新的基于单细胞~RNA-seq~数据的基因调控网络构建方法~scGRNHunter。
识别细胞类型特征和细胞的基因表达活动程序~(如生命周期过程、对环境因素的反应)~对于理解细胞和组织的组成至关重要,
,也有助于了解基因间的调控机制。
虽然单细胞~RNA-Seq~可以量化成个体细胞中的转录本,
每个细胞的表达谱可能是这两种类型的程序的混合物,使它们难以分离。
在这里,我们提出了一个使用矩阵分解的算法~WSSMFA~来解决这个问题。
通过在公开的大脑类器官~scRNA-Seq~数据集上的实验表明,我们提出的~scGRNHunter~方法可以准确地推断出身份和活动性的子程序, 
并在此基础上构建成基于细胞类别身份的基因调控网络和基于细胞活动的基因调控网络。

% 图X幅,表X个,参考文献X篇(四号宋体)

\end{abstractcn}