%!TEX root = ../csuthesis_main.tex
% 设置中文摘要
\keywordscn{基因调控网络; 单细胞~RNA-seq~数据; 互信息; 动态网络构建; 细胞异质性;}
\categorycn{TP391}
\itemcountcn{图 \totalfigures\ 幅,表 \totaltables\ 个,参考文献 \total{citnum}\ 篇}
% \addcontentsline{toc}{section}{摘要}
\begin{abstractcn}\setlength{\baselineskip}{20pt}%\renewcommand{\baselinestretch}{1.0}
基因调控网络~(GRNs)~是基因之间的相互作用所形成的网络,揭示了基因中复杂的调控关系。
构建基因调控网络有助于我们了解基因调控机制、生物网络系统所蕴涵的结构和功能信息,
从分子水平上理解肿瘤等复杂疾病发生的机理。
基因调控网络的构建是系统生物学中的一个最核心也是最重要的问题。
针对基因调控网络推断研究中准确度低,网络稀疏等难点, 
本文在~DNA~微阵列和单细胞~RNA-seq~测序技术产生的数据集上,提出了有效的基因调控网络构建方法。
本文的创新点和主要工作如下:

(1)~针对当前基于互信息的方法构建准确度低的问题,
本文提出了一种基于互信息和局部结构的基因调控网络构建方法~Loc-PCA-CMI。
该方法根据基因的共表达关系来识别局部重叠基因簇,并基于条件互信息~(PCA-CMI)~的路径一致性算法推断每个基因簇的局部网络结构,
最终通过聚合局部网络结构,来确定最终的基因调控网络。
本文在~DREAM3~敲除数据集上对~LOC-PCA-CMI~进行了评估,
将其性能与其它四种基于信息理论的网络结构推断方法进行了比较。
特别是在基因数目为~50~和~100~的网络上的实验结果表明,
~Loc-PCA-CMI~在~DREAM3~数据集上降低了构建中冗余的依赖关系,
在~AUPR~上表现优于其它四种基准方法。

(2)~针对当前基于数据驱动方法无法构建全局网络的缺陷,
本文提出了一种数据驱动的基因调控网络构建方法~D3GRN。
该方法使用了抽样的策略~(bootstrapping),
将每个目标基因的调控关系转化为函数分解问题并利用揭示网络相互作用的算法解决各个子问题。
为了弥补数据驱动方法无法构建全局网络的缺陷,
本文采用了抽样策略和基于面积的评分方法来推断最终的网络。
在~DREAM4~和~DREAM5~基准数据集上的实验结果表明,
~D3GRN~在~AUPR~这个评价指标上与其它基准方法相比具有竞争力。

(3)~针对当前在单细胞~RNA-seq~数据集上细胞聚类不准确的问题,
本文提出了一种基于随机森林相似性学习的单细胞细胞聚类方法~RafClust。
该方法使用多种相关性度量方法来刻画细胞的特征, 
并使用随机森林回归模型进一步学习细胞与细胞之间的相似性矩阵,
基于相似性矩阵后采用层次聚类来决定细胞的最终类别。
在十个单细胞数据集上的实验结果表明,~RafClust~在~ARI~上表现优于其它六种基准方法。

(4)~针对当前从超大规模的单细胞~RNA-seq~数据中识别稀有细胞的算法非常耗时或耗费内存的问题,
本文提出了一种基于孤立森林的单细胞稀有细胞识别方法~DoRC。
该方法利用孤立森林高效地来对每个细胞产生稀有度分数,
结合阈值方法对细胞进行稀疏性的二元标注。
在超大规模的单细胞~RNA-seq~数据~${\sim}68$k~人血细胞的单细胞表达谱上的实验结果表明,
~DoRC~在划分人类血液树突状细胞亚型方面有突出的效果,执行效率高。
另外,~DoRC~可以识别仿真数据集里面的稀有细胞,并且对细胞类型特征也很敏感。

(5)~针对当前从单细胞~RNA-seq~数据中无法同时推断出与细胞类型相关和与细胞活动相关的基因调控网络的问题,
本文提出了一种基于矩阵分解的基因调控网络构建方法~scGRNHunter。
本文首先提出了矩阵分解算法~WSSMFA~在单细胞~RNA-seq~数据上同时分离出细胞类型程序和细胞活动程序,
在此基础上结合公开数据库~TRRUST~构建基于每个程序的基因调控网络。
在公开的大脑类器官~scRNA-Seq~数据集上的实验结果表明,
本文提出的~scGRNHunter~方法可以有效推断出身份和活动性的子程序, 
并在此基础上构建基于细胞类型的基因调控网络和基于细胞活动的基因调控网络。

\end{abstractcn}