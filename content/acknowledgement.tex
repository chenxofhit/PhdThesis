\section{致谢} % 无章节编号

六年的博士生涯就要画上一个句点,行笔至此,黯然神伤不能自已。
回想自己从程序员转变为科研工作者,
从师弟转变为师兄,
从单身转变为两个孩子的父亲,
沧海变桑田,仿佛过了很久,恍惚又觉得是刹那之间。
这六年的研究生涯,也许是我人生之旅压力最大、自我反思最为频繁的时段,
幸好有幸遇到了各位良师益友,和家人的陪伴,
常思奋不顾身以全力以赴,然而个人能力实在有限,但是只取得了一点点成绩,也留下很多遗憾。
博士阶段从本质上讲是一种不断思考、不断妥协、不断前进的生活状态,
学位只在其表,
博士虽然即将结束了但这种状态将会常伴一生。

首先要万分感谢我的导师李敏教授,亦师亦友。
六年里面耳濡目染,学术上严谨、认真,有目标,有计划,敢想敢执行;
工作上认真负责,一丝不苟,不骄不躁;
生活上待人接物温文尔雅,让人如沐春风。
论文的最初的选题立意和旷日持久的实验,离不开李老师的谆谆教导。
生活上李老师给予了我极大的帮助,排忧解难,让我安心科研。
祝李老师万事如意、身体健康,永远美丽,永远幸福。
十分感谢王建新教授、潘毅教授、吴方向教授、王伟平教授、黎耀航教授。
王建新老师学术渊博,潘毅老师幽默风趣,吴方向老师严谨认真,王伟平老师知性睿智,黎耀航老师亲和有为。
他们在学术上功力深厚,看问题一针见血,散发着独特的人格魅力。
高山仰止,景行行止,虽不能至,然心向往之。
感谢丁小军老师、钟坚成老师、罗军伟老师、罗慧敏老师、彭晓清老师、刘锦老师、李洪东老师,
感谢您们对我生活和学业上的无私帮助和指导,祝您们步步高升。
感谢段桂华老师、盛羽老师,感谢您们这些年对实验室的支持。

其次我要感谢生物组这个大家庭,
感谢~213、216~实验室的各位同学,六年时间遇到了太多值得学习的师兄师姐师弟和师妹,
他们身上闪烁着对知识的渴望,对生活的热爱和对自我的不断超越,点点滴滴时刻让我深受鼓舞、不敢懈怠。
同学名单很长,但恐有遗漏,故不在这里详细罗列,转而在网络上维护一份列表\footnote{https://github.com/chenxofhit/PhdThesis/tree/master/acknowledgement.txt}。
在他们身上我学到了许多,祝你们学业和事业心想事成,幸福美满。

再次要感谢我的父母、岳父岳母、妻子、哥哥嫂子,还有两个可爱的女儿。
身体发肤受之于父母,父母的爱如高山流水长;
岳父岳母是再生父母,他们的恩情深沉而博大。
四位老人年逾天命,两鬓霜凝,六年时间里,间断帮我抚养小孩,接济不时之需,
观父母容颜渐改,垂垂老矣,我狠未能建功立业,每每念及至此羞愧难当。
结草衔环,当终身不忘。
感谢妻子的默默陪伴,与我组建了一个美满的家庭,
让你受了很多苦,有时候惹你生气,大部分时候是我错了。
希望一如既往,携手前进,白头偕老。
哥哥和嫂子远在深圳,这六年间替我扛下了父母赡养之事代行子女之孝,
在生活上和思想上给了我诸多帮助和启发,谢谢你们的支持和付出!
还要谢谢我的两个女儿陈康羲和陈康和,你们的降临给家庭带来了无穷的欢乐和喜悦,
也祝你们永远幸福快乐健康。

最后感谢中南大学给了我这样的学习机会, 
让我在一流的平台里遇到诸多良师益友,
安心学习和享受生活。
祝愿中南大学计算机学院在王建新院长的带领下, 
在诸位老师的辛勤耕耘下,桃李满天下,更上一层楼。
在我的本科和硕士论文的致谢末尾都有一句海子的诗,在此也献给读到这里的每一个人:
愿你有一个灿烂的前程,愿你有情人终成眷属,愿你在尘世获得幸福。