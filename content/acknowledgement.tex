\section{致谢} % 无章节编号

行笔至此花了整整六年的时间,博士生涯即将画上句点,黯然神伤不能自已。
回想自己从程序员转变为科研工作者,
从师弟转变为师兄,
从单身转变为两个孩子的父亲,
沧海变桑田,仿佛过了很久,恍惚又觉得是刹那之间。
这六年的研究生涯,是我目前人生之旅压力最大、自我反思最为频繁的时段。
多少次实验结果不理想, 深夜在家门口徘徊, 屋里的灯光亮着, 两个小孩子早已沉入梦境。
有时候想放弃,但是既然选择了学术这条道路还是不能自我放逐。
有幸遇到了各位良师益友和家人默默的陪伴,
生活固然不美好, 我也是在枷锁之中力求能发光发热贡献自己的一份力量,
常思奋不顾身,个人能力实在有限,只取得了一点点成绩,也留下很多遗憾。
这一特殊阶段从本质上讲是一种从各方面不断思考、不断妥协、不断前进的生活状态,
学位只在其表,
博士阶段虽然即将结束了但这种状态和激情将会常伴一生。

首先要十分感谢我的导师李敏教授,亦师亦友。
六年里面耳濡目染,学术上严谨、认真,有目标,有计划,敢想敢执行;
工作上认真负责,一丝不苟,不骄不躁;
生活上待人接物温文尔雅,让人如沐春风。
博士研究方向的选题立意,
各个细分实验的讨论、对应的英文论文地撰写投稿和返修,
到最后本文的撰写和反反复复地修订,离不开李老师的谆谆教导和不断鼓励。
生活上李老师给予了我极大的帮助,排忧解难,让我安心科研。
祝李老师万事如意、身体健康,永远美丽,永远幸福。
感谢王建新教授、潘毅教授、吴方向教授、王伟平教授、黎耀航教授。
王建新老师以身作则为人师表,潘毅老师幽默风趣,吴方向老师严谨认真,王伟平老师知性睿智,黎耀航老师亲和有为。
您们学识渊博,探讨学术问题一针见血,各自散发出独特的人格魅力。
高山仰止,景行行止,虽不能至,然心向往之。

感谢丁小军老师、钟坚成老师、罗军伟老师、罗慧敏老师、彭小清老师、刘锦老师、李洪东老师,
感谢您们对我生活和学业上的无私帮助和指导,祝您们步步高升。
感谢段桂华老师、盛羽老师,感谢您们这些年对实验室的支持。
感谢国外的 Dr.Divyanshu Talwar, Dr.Atul Deshpande, Dr.Anthony Gitter, Dr.David Yanni,
 Dr.-Ing.Saquib Sarfraz, Dr.Shahin Mohammadi, Dr.Aashi Jindal, Dr.Debajyoti Sinha, Dr.Ulysse Herbach, Dr.Annamalai Muthiah, Dr.Jose Luis Casadiego Bastidas
等人对本文里面涉及到的方法和数据集等方面的讨论和指点,与他 (她) 们的交流让我受益匪浅。


其次我要感谢生物组这个大家庭,
感谢 213、216 实验室的各位同学,六年时间遇到了太多值得学习的师兄师姐师弟和师妹,
您们身上闪烁着对知识的渴望,对生活的热爱和对自我的不断超越,点点滴滴时刻让我深受鼓舞、不敢懈怠。
师兄师姐师弟和师妹的名单比较长,目前共计四十余人 (以姓氏首字母排序): 房森彪、冯浩楠、高昊、郭林沅、胡昕昱、黄兰、蒋辉、兰伟、李幸一、梁珍兰、刘亮亮、卢长利、刘澜、刘丽娟、李一鸣、孟祥茂、
倪鹏、任立男、尚娟、沈曦、唐丽、田宇、王林从花、王荣、项炬、徐紫薇、余颖、杨梦云、严承、 杨洁、杨昌获、张振、郑瑞清、朱凌志、曾敏、张燕、张富豪、张文静、张佳帅、赵凯杰,
顾此失彼恐有遗漏,故也在网络上维护一份列表\footnote{https://github.com/chenxofhit/PhdThesis/tree/master/acknowledgement.txt}作为补充。
在您们身上我学到了许多,祝您们学业和事业心想事成,幸福美满。

再次要感谢我的父母、岳父岳母、妻子、哥哥嫂子,还有两个可爱的女儿。
身体发肤受之于父母,父母的爱如高山流水长;
岳父岳母是再生父母,您们的恩情深沉而博大。
四位老人年逾天命,两鬓霜凝,六年时间里,间断帮我抚养小孩,接济不时之需,
观父母容颜渐改,垂垂老矣,我狠未能建功立业,每每念及至此羞愧难当。
结草衔环,当终身不忘。
感谢妻子的默默陪伴,与我组建了一个美满的家庭,
让你受了很多委屈,有时候惹你生气,
希望一如既往,携手前进,白头偕老。
哥哥和嫂子远在深圳,这六年间替我扛下了父母赡养之事,
在生活上和思想上给了我诸多帮助和启发,谢谢您们的支持和付出!
还要谢谢我的两个女儿陈康羲和陈康和,你们的降临和成长给家庭带来了无穷的欢愉和喜悦,
也祝你们永远幸福快乐健康。

再次万分感谢中南大学给了我这样的学习机会, 
让我在一流的平台里遇到诸多良师益友,
并能安心生活和潜心学习。
祝愿中南大学计算机科学与技术学院在王建新院长的带领下, 
在诸位老师的辛勤耕耘下,桃李满天下,更上一层楼。
感谢 R、Python、\LaTeX、Inkscape、Git、GNU Linux 等开源软件和 MATLAB、OminiGraffle、Visual Studio Code、Microsoft Powerpoint 等商业软件的
支撑,本文的实验、作图、撰写、版本管理均离不开这些软件。
在我的本科和硕士论文的致谢末尾都有一句海子的诗,在此也和拙著一起献给读到这里的每一个人:
愿你有一个灿烂的前程,愿你有情人终成眷属,愿你在尘世获得幸福。

最后,衷心感谢各位专家在百忙之中对本论文进行审阅和指导!