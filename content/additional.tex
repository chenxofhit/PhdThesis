%!TEX root = ../csuthesis_main.tex
{~}
\vspace{-9pt}
\section{攻读学位期间主要研究成果} % 无章节编号

\ifblindreview
% \noindent
% (盲审隐去作者相关具体信息)
\fi
\vspace{11pt}
\subsection*{一、学术论文}

\ifblindreview

% 学位办老师要求用如下这种几乎算是单盲的格式,我也木有办法……
\subsubsection*{已录用/检索论文}
%\noindent
第一作者:
\begin{enumerate}[label={[\arabic*]},itemindent=2em,wide]

\item A novel method of gene regulatory network structure inference from gene knock-out expression data[J]. Tsinghua Science and Technology 24, no. 4 (2019): 446-455.{\bfseries \heiti(SCI 检索,JCR 3 区)} 
\item D3GRN: a data driven dynamic network construction method to infer gene regulatory networks[J]. BMC genomics 20, no. 13 (2019): 1-8.{\bfseries \heiti(SCI 检索,JCR 1 区)} 
\item DoRC: Discovery of rare cells from ultra-large scRNA-seq data[C]. In 2019 IEEE International Conference on Bioinformatics and Biomedicine (BIBM), pp. 111-116. IEEE, 2019.{\bfseries \heiti(EI 检索)}

\end{enumerate}
%第二作者:
%\begin{enumerate}[label={[\arabic*]},itemindent=2em,wide]
%\item CSU Latex Template[J]. CSU player: 1(1):1-10. {\bfseries \heiti(SCI 检索,JCR 1 区)}
%\item CSU Latex Template[J]. CSU player: 1(1):1-10. {\bfseries \heiti(SCI 检索,JCR 2 区)}
%\end{enumerate}
%\noindent
\subsubsection*{投稿状态论文}
%\noindent
第一作者:
\begin{enumerate}[label={[\arabic*]},itemindent=2em,wide]
\item scGRNHunter: a gene regulatory network reconstruction method from single cell RNA-seq data[J]. Bioinformatics. {\bfseries \heiti(SCI JCR 1 区)}
\end{enumerate}
%第二作者:
%\begin{enumerate}[label={[\arabic*]},itemindent=2em,wide]
%\item CSU Latex Template. XXX Transactions on CSU player. {\bfseries \heiti(SCI Under Review,JCR 1 区)}
%\item CSU Latex Template. XXX Transactions on CSU player. {\bfseries \heiti(SCI Under Review,JCR 2 区)}
%\end{enumerate}


\else
% 标准版本
\begin{enumerate}[label={[\arabic*]},itemindent=2em,wide]

\item \textbf{Xiang Chen}, Min Li, Ruiqing Zheng, Siyu Zhao, Fang-Xiang Wu, Yaohang Li, and Jianxin Wang. A novel method of gene regulatory network structure inference from gene knock-out expression data. Tsinghua Science and Technology 24, no. 4 (2019): 446-455.{\bfseries \heiti(SCI 检索,JCR 2 区)} 
\item \textbf{Xiang Chen}, Min Li, Ruiqing Zheng, Fang-Xiang Wu, and Jianxin Wang. D3GRN: a data driven dynamic network construction method to infer gene regulatory networks. BMC genomics 20, no. 13 (2019): 1-8.{\bfseries \heiti(SCI 检索,JCR 1 区)} 
\item \textbf{Xiang Chen}, Fang-Xiang Wu, Jin Chen, and Min Li. DoRC: Discovery of rare cells from ultra-large scRNA-seq data. In 2019 IEEE International Conference on Bioinformatics and Biomedicine (BIBM), pp. 111-116. IEEE, 2019.{\bfseries \heiti(EI 检索, CCF B~类推荐国际会议)}
\item \textbf{Xiang Chen}, Min Li, Ruiqing Zheng, Fang-Xiang Wu, and Jianxin Wang. scGRNHunter: a gene regulatory network reconstruction method from single cell RNA-seq data. Bioinformatics. {\bfseries \heiti(拟投稿)}
\item Data-driven clustering recommendation method for single-cell RNA-sequencing data.Yu Tian, Ruiqing Zheng, Zhenlan Liang, {\bf Xiang Chen}, Suning Li, Min Li. IEEE/ACM Transactions on Computational Biology and Bioinformatics.{\bfseries \heiti(投稿中)}
\item Hui Jiang, Mengyun Yang, \textbf{Xiang Chen}, Min Li, Yaohang Li, and Jianxin Wang. miRTMC: A miRNA Target Prediction Method Based on Matrix Completion Algorithm[J]. IEEE JOURNAL OF BIOMEDICAL AND HEALTH INFORMATICS. 2020, DOI:10.1109/JBHI.2020.2987034. {\bfseries \heiti(SCI 检索,JCR 1 区)}
\item Ruiqing Zheng, Zhenlan Liang, \textbf{Xiang Chen}, Yu Tian, Min Li. An Adaptive Sparse Subspace Clustering for Cell Type Identification. Front Genet. 2020 Apr 28;11:407. doi: 10.3389/fgene.2020.00407.{\bfseries \heiti(SCI 检索,JCR 2 区)}
\item Ruiqing Zheng, Min Li, \textbf{Xiang Chen}, Fang-Xiang Wu, Yi Pan, and Jianxin Wang. BiXGBoost: a scalable, flexible boosting-based method for reconstructing gene regulatory networks. Bioinformatics 35, no. 11 (2019): 1893-1900. {\bfseries \heiti(SCI 检索,JCR 1 区)}
\item Ruiqing Zheng, Min Li, \textbf{Xiang Chen}, Siyu Zhao, Fang-Xiang Wu, Yi Pan, Jianxin Wang. An ensemble method to reconstruct gene regulatory networks based on multivariate adaptive regression splines. IEEE/ACM Transactions on Computational Biology and Bioinformatics. DOI: 10.1109/TCBB.2019.2900614  {\bfseries \heiti(SCI 检索,JCR 1 区)}

\item 王荣,王建新,\textbf{陈向},盛羽.面向异构资源集成的数字图像实验平台[J].计算机工程与应用 54, no.15 (2018):185-191.{\bfseries \heiti(EI 期刊)}
\end{enumerate}
\fi

\vspace{22pt}
\subsection*{二、主持和参与的科研项目}
\ifblindreview
\else
\begin{enumerate}[label={[\arabic*]},itemindent=2em,wide]

    \item 国家自然科学基金优秀青年项目: 生物信息学, 项目编号:61622213,参与。

\end{enumerate}

\fi

\newpage

\ifblindreview
\else
% 由于 acknowledgement 比较长,单独使用一个文件来汇总。FIXME:chenxofhit@gmail.com
\section{致谢} % 无章节编号

六年的博士生涯就要画上一个句点,行笔至此,黯然神伤不能自已。
回想自己从程序员转变为科研工作者,
从师弟转变为师兄,
从单身转变为两个孩子的父亲,
沧海变桑田,仿佛过了很久,恍惚又觉得是刹那之间。
这六年的研究生涯,也许是我人生之旅压力最大、自我反思最为频繁的时段,
幸好有幸遇到了各位良师益友,和家人的陪伴,
常思奋不顾身以全力以赴,然而个人能力实在有限,但是只取得了一点点成绩,也留下很多遗憾。
博士阶段从本质上讲是一种不断思考、不断妥协、不断前进的生活状态,
学位只在其表,
博士虽然即将结束了但这种状态将会常伴一生。

首先要万分感谢我的导师李敏教授,亦师亦友。
六年里面耳濡目染,学术上严谨、认真,有目标,有计划,敢想敢执行;
工作上认真负责,一丝不苟,不骄不躁;
生活上待人接物温文尔雅,让人如沐春风。
论文的最初的选题立意和旷日持久的实验,离不开李老师的谆谆教导。
生活上李老师给予了我极大的帮助,排忧解难,让我安心科研。
祝李老师万事如意、身体健康,永远美丽,永远幸福。
十分感谢王建新教授、潘毅教授、吴方向教授、王伟平教授、黎耀航教授。
王建新老师学术渊博,潘毅老师幽默风趣,吴方向老师严谨认真,王伟平老师知性睿智,黎耀航老师亲和有为。
他们在学术上功力深厚,看问题一针见血,散发着独特的人格魅力。
高山仰止,景行行止,虽不能至,然心向往之。
感谢丁小军老师、钟坚成老师、罗军伟老师、罗慧敏老师、彭晓清老师、刘锦老师、李洪东老师,
感谢您们对我生活和学业上的无私帮助和指导,祝您们步步高升。
感谢段桂华老师、盛羽老师,感谢您们这些年对实验室的支持。

其次我要感谢生物组这个大家庭,
感谢~213、216~实验室的各位同学,六年时间遇到了太多值得学习的师兄师姐师弟和师妹,
他们身上闪烁着对知识的渴望,对生活的热爱和对自我的不断超越,点点滴滴时刻让我深受鼓舞、不敢懈怠。
同学名单很长,但恐有遗漏,故不在这里详细罗列,转而在网络上维护一份列表\footnote{https://github.com/chenxofhit/PhdThesis/tree/master/acknowledgement.txt}。
在他们身上我学到了许多,祝你们学业和事业心想事成,幸福美满。

再次要感谢我的父母、岳父岳母、妻子、哥哥嫂子,还有两个可爱的女儿。
身体发肤受之于父母,父母的爱如高山流水长;
岳父岳母是再生父母,他们的恩情深沉而博大。
四位老人年逾天命,两鬓霜凝,六年时间里,间断帮我抚养小孩,接济不时之需,
观父母容颜渐改,垂垂老矣,我狠未能建功立业,每每念及至此羞愧难当。
结草衔环,当终身不忘。
感谢妻子的默默陪伴,与我组建了一个美满的家庭,
让你受了很多苦,有时候惹你生气,大部分时候是我错了。
希望一如既往,携手前进,白头偕老。
哥哥和嫂子远在深圳,这六年间替我扛下了父母赡养之事代行子女之孝,
在生活上和思想上给了我诸多帮助和启发,谢谢你们的支持和付出!
还要谢谢我的两个女儿陈康羲和陈康和,你们的降临给家庭带来了无穷的欢乐和喜悦,
也祝你们永远幸福快乐健康。

最后感谢中南大学给了我这样的学习机会, 
让我在一流的平台里遇到诸多良师益友,
安心学习和享受生活。
祝愿中南大学计算机学院在王建新院长的带领下, 
在诸位老师的辛勤耕耘下,桃李满天下,更上一层楼。
在我的本科和硕士论文的致谢末尾都有一句海子的诗,在此也献给读到这里的每一个人:
愿你有一个灿烂的前程,愿你有情人终成眷属,愿你在尘世获得幸福。
\newpage
\fi