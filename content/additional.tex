\section{攻读学位期间主要研究成果} % 无章节编号

\ifblindreview
\noindent
(盲审隐去作者相关具体信息)
\fi
\subsection*{学术论文}

\ifblindreview

\noindent
第一作者:JCR 1 区 x 篇,会议 x 篇 \\{}
第二作者:JCR 1 区 x 篇,3 区 x 篇,4 区 x 篇,EI x 篇 

\noindent
投稿状态: 
IEEE Transactions on Image Processing 1篇(under review)\\{}
IEEE Transactions on Circuits and Systems for Video Technology 1 篇(Accept with Minor Revision)
\else
% 标准版本
\begin{enumerate}[label={[\arabic*]}]
\item \textbf{Daxia Guo}, Director, Someone. CSU Latex Template[J]. CSU player: 1(1):1-10. (SCI 检索,JCR 1 区)
\item Director, \textbf{Daxia Guo}, Someone, Someother. XXXXXX[J]. Transactions on Image Processing.(SCI Under Review,JCR 1 区)
\item Director, \textbf{Daxia Guo}, Someone, Someother. XXXXXX[J]. Transactions on Circuits and Systems for Video Technology.(SCI Under Review,JCR 1 区)
\end{enumerate}
\fi
\subsection*{发明专利}
\ifblindreview
\noindent
发明专利 2 项,已公开
\else
\begin{enumerate}[label={[\arabic*]}]
\item 郭大侠,XXX,XXX. 一种用Latex写中南大学学位论文的方法. 申请号:CN20190415xxxx,公开号:CNXXXXXXXXXA
\end{enumerate}
\fi

\ifblindreview
\else
\subsection*{主持和参与的科研项目}
\begin{enumerate}[label={[\arabic*]}]
\item 国家自然科学基金面上项目《XXXXXXXXXXXX》, 项目编号:XXXXXXXX,参与.
\end{enumerate}

\subsection*{个人获奖情况}
\noindent
\begin{enumerate}[label={[\arabic*]}]
\item XX金奖
\item XX奖学金
\end{enumerate}
\fi

\newpage

\ifblindreview
\else
\section{致谢} % 无章节编号
六年的博士生涯就要画上一个句点,行笔至此,黯然神伤不能自已。
回想自己从程序员转变为科研工作者,
从师弟转变为师兄,
从单身转变为两个孩子的父亲,
斗转星移,仿佛过了很久,恍惚又觉得是刹那之间。
这六年的研究生涯,也许是我人生之旅压力最大、自我反省最为频繁的时段,
幸好有幸遇到了各位良师益友,和家人的陪伴,
常思奋不顾身以全力以赴,然而个人能力实在有限,但是只取得了一点点成绩,也留下很多遗憾。
博士阶段从本质上讲是一种不断思考不断妥协不断前进的生活状态,
学位只在其表,
博士虽然即将结束了但这种状态将会常伴一生。

首先要感谢我的导师李敏教授,亦师亦友,有时还是家人。
六年里面耳濡目染,学术上严谨、认真,有目标,有计划,敢想敢执行;
工作上除了认真负责一丝不苟,遇到任何问题第一时间想解决方案;
生活上待人接物让人如沐春风。
祝我的导师万事如意,身体健康,永远美丽。
感谢王建新教授,潘毅教授,吴方向教授,学术上的独到的见解和个人独特的人格魅力,无处不在,深受浸淫。
高山仰止,景行行止,虽不能至,然心向往之。
感谢钟坚成老师,罗军伟老师,罗慧敏老师,彭晓清老师,刘锦老师,李洪东老师,
感谢你们对我生活和学业上的无私帮助和指导,祝您们步步高升。
感谢段桂华老师,盛羽老师,感谢您们这些年对实验室的支持。

其次我要感谢生物组这个大家庭,
感谢~213、216~实验室的各位同学,六年时间遇到了太多值得学习的师兄师姐师弟和师妹,
他们身上闪烁着对知识的渴望,对生活的热爱和对自我的不断超越,点点滴滴让我深受鼓舞、不敢懈怠。
名单会有很长,但是恐有遗漏,故不在这里详细罗列,转而在网上维护一份动态列表。
在他们身上我学到了许多,祝你们学业和事业心想事成,幸福美满。

再次要感谢我的父母,我的岳父岳母,我的妻子,我的哥哥嫂子,还有我的两个可爱的女儿。
身体发肤受之于父母,父母的爱如高山流水长。
岳父岳母是再生父母,他们的恩情深沉而博大。
四位老人年逾天命,两鬓霜凝,六年时间里,间断帮我抚养小孩,接济不时之需,
观父母容颜渐改,垂垂老矣,我狠未能建功立业,每每念及至此羞愧难当。
结草衔环,当终身不忘。
感谢妻子的默默陪伴,与我组建了一个美满的家庭,
让你受了很多苦,有时候惹你生气,大部分时候是我错了。
希望一如既往,携手前进,白头偕老。
哥哥和嫂子远在深圳,这六年间替我扛下了父母赡养之事代行子女之孝,
在生活上和思想上给了我诸多帮助和启发,谢谢你们的支持和付出!
还要谢谢我的两个女儿陈康羲和陈康和,你们的降临给家庭带来了无穷的欢乐和喜悦,
也祝你们永远幸福快乐健康。

最后感谢中南大学给了我这样的学习机会, 
让我在一流的平台里遇到诸多良师益友,
安心学习和享受生活。
祝愿中南大学计算机学院在王建新院长~(教授)~的带领下, 
在诸位老师的辛勤耕耘下,桃李满天下,更上一层楼。
在我的本科和硕士论文的致谢末尾都有一句海子的诗,在此也献给读到这里的每一个人:
愿你有一个灿烂的前程,愿你有情人终成眷属,愿你在尘世获得幸福。
\newpage
\fi