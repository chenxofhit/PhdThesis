\section{一种新的基于基因敲除表达数据的基因调控网络结构推断方法}

从基因表达数据推中推断基因调控网络~(GRN)~的结构一直是系统生物学中的十分具有挑战的问题。
鉴定基因之间复杂的调控关系对于理解细胞的调控机制至关重要。
至今基于信息理论的各种~GRN~推断方法已经被提出来。
然而,由于原始数据中存在的外部噪声、
网络结构中的拓扑稀疏性和非线性基因之间的依赖等因素,
这些方法在网络推断中会引入冗余的依赖关系。
特别是随着网络规模的增加,这些方法的表现大幅降低。
在本章节中我们提出了一种新的网络结构推断方法~Loc-PCA-CMI:
首先识别局部重叠基因簇,
然后基于条件互信息~(PCA-CMI)~的路径一致性算法推断每个簇的局部网络结构,
最终通过聚合局部网络结构,也就是基因之间的依赖性网络,来构造最终的~GRN。
我们在~DREAM3~敲除数据集上对~LOC-PCA-CMI~进行了评估,
将其性能与其它基于信息理论的网络结构推断方法,包括~ARACNE、MRNET、PCA-CMI~和~PCA-PMI,进行了比较。
实验结果证明,~Loc-PCA-CMI~在~DREAM3~数据集上特别是在基因数目为~50~和~100~的网络上表现优于其它四种基准方法。

\subsection{介绍}

推断和理解基因调控网络~(GRN)~是系统生物学中的一个关键问题, 
可以帮助生物医学科学家明确识别基因与基因之间复杂的调控关系、理解细胞中的调控机制~\cite{altay2010inferring, basso2005reverse}。
在过去,GRN~是从实验干预中推断出来的,其中基因之间的调控相互作用被验证。
显然,这种方法是不可行的~\cite{elnitski2006locating},需要耗费大量时间和相当大的成本。
由于微阵列技术的发展,大量的基因表达数据通过测序被得到,这使得基于计算方法从这些表达数据中推断出~GRN~成为可能~\cite{maetschke2013supervised}。
近年来,基于计算方法的网络推断已成为最重要的目标之一~\cite{altay2010inferring, margolin2006reverse}。
已经提出了各种用于GRN推断的方法,例如基于回归的方法~\cite{Huynh-Thu2010, Haury2012, Huynh-Thu2014, liu2014group, li2017mgt, zheng2018bixgboost},
基于微分方程的方法~\cite{sakamoto2001inferring, chowdhury2015stochastic, li2011large},
贝叶斯和动态贝叶斯网络~\cite{murphy1999modelling, zou2004new, vinh2011globalmit, young2014fast, Liu2016, omranian2016gene},
以及基于状态空间的方法~\cite{wu2003modeling, quach2007estimating}。
不幸的是,基因表达数据通常具有高维度和相对较小的样本量,存在“维度诅咒”的问题~\cite{wang2006inferring}。
此外,基因表达数据通常涉及大量外部噪声和非线性关系。
所有这些问题使得准确推断基因之间的调控相互作用,
尤其是在后基因组时代处理大规模基因表达数据时,
变得更加复杂和具有挑战性。

如果不考虑基因之间的上游或下游调控关系,
并且忽略自调控机制,~GRN~可以用无向非循环图来表示,
其中每个节点对应于一个基因,而每条边代表基因之间的调控关系。
人们基于各种不同的假设和不同的条件提出了从表达数据构建~GRN~精确结构的各种不同的计算方法~\cite{longabaugh2005computational,karlebach2008modelling}。
目前的这些方法可以大致分为基于模型和无模型两大类别。

基于模型的方法通常制定系统的计算模型并进一步学习这种模型的参数。
典型的计算模型包括布尔网络~\cite{shmulevich2002probabilistic,kim2007boolean,bornholdt2008boolean,zhou2016relative},
贝叶斯网络~\cite{kim2003inferring,zou2004new,chen2006effective,needham2007primer,lo2015high},
以及微分方程模型~\cite{gardner2003inferring,di2005chemogenomic,bansal2006inference, honkela2010model,lu2011high,li2011large}。
布尔网络模型是最简单的网络模型,它通过布尔变量和布尔逻辑实现。
因为基因表达的状态被认为只是活动或非活动,布尔网络模型不能完全捕获复杂的系统行为~\cite{lee2009computational}。
贝叶斯网络模型是一种流行的概率图形模型,其中基因之间的依赖关系通过有向无环图描述。
贝叶斯网络模型在处理噪声和结合先验知识方面优于其它模型,
但该模型中的结构学习是计算密集型的,并且已被证明是~NP~难问题~\cite{chickering2004large}。
微分方程模型通过函数表征基因在特定时间的表达水平,其涉及与其它基因的调控相互作用。
通常,它将一个基因表达的变化率~(导数)~作为其它相关基因表达水平的函数。
使用微分方程模型重建~GRN~的一个主要挑战是如何在高维模型中有效识别模型结构和估计参数。
关于各种数据驱动建模方案和相关主题的文章评论,可以参考~\cite{hecker2009gene,marbach2012wisdom,wu2007inference,liu2012reverse,li2018control}。

除了基于模型的方法外,无模型方法主要通过衡量基因之间的依赖性来识别调控相互作用。
典型的算法包括基于相关性和基于信息理论的方法。
在基于相关性的方法中,调控相互作用由两个基因之间的共表达程度决定,
例如~Pearson~相关性,秩相关性,欧几里德距离和一对表达值向量之间的角度~\cite{wang2014review}。
然而,基于相关性的方法无法识别基因之间的复杂依赖性,例如非线性依赖性~\cite{ruyssinck2014nimefi}。
此外,相当多的功能相关基因可能不会共表达,这使得难以准确推断调控相互作用。
基于信息理论的方法也是一种代表性的无模型方法,其中互信息~(MI)~有利于衡量基因之间的潜在依赖性,
因为它可以有效地捕获非线性依赖关系~\cite{brunel2010miss,zhang2011inferring}。
近年来,已经提出了基于信息论的各种网络推断方法,
其侧重于区分直接调控相互作用与间接关联~\cite{marbach2010revealing}。
为消除间接相互作用,
Margolin~等人~\cite{margolin2006aracne}提出了基于数据处理不等式的~ARACNE~方法,并考虑了交互三角传递性质。
Meyer~\cite{meyer2007information}~的最小冗余网络~(MRNET)~使用最小冗余特征选择方法~\cite{peng2005feature},
其中对于网络中的每个候选基因,它选择其高度相关基因的子集,同时最小化所选基因之间基于互信息的标准。
Zhang~等人~\cite{zhang2011inferring}介绍了一种基于条件互信息~(PCA-CMI)~的路径一致性算法; 
Zhao~等人~\cite{zhao2016part}引入了基于偏互信息~(PCA-PMI)~的路径一致性算法。
路径一致性算法~(PCA)~是一种穷举算法,广泛用于推断~GRN~\cite{zhang2011inferring}。
~PCA-CMI~和~PCA-PMI~这两个算法通常会在运行时间和准确度之间进行折中权衡。
随着网络规模的增加,复杂网络结构里不可控制的外部噪声使得~GRN~的预测精度急剧下降。
为了改善这种情况,利用分而治之的策略,我们首先使用头部的高度共同表达的基因作为局部聚类的质心; 
然后用~PCA-CMI~对每个簇的精确结构进行细化。
然后,~GRN~的最终结构与所有局部网络结构的集合一起推断。
我们将这种新方法命名为~Loc-PCA-CMI。
从直觉上也可以看出,~Loc-PCA-CMI~方法可以处理相对较大的数据集,
并且受益于~PCA-CMI~在小尺寸基因子网的相对准确的结构推断。

\subsection{方法}
在本节中,我们将介绍信息理论中的相关工作,包括熵、互信息~(MI)和条件互信息~(CMI),
和我们提出的~GRN~结构推断方法~Loc-PCA-CMI。


\subsubsection{相关工作}
利用测量两个变量之间的非线性依赖关系时相对高效的优点,
信息理论越来越多地用于衡量基因间的调控关系强弱。
互信息~(MI)的定义如下:
\begin{align} % requires amsmath; align* for no eq. number
    MI(X,Y)=\int \int p(x,y)log \frac{p(x,y)}{p(x)p(y)}dxdy
 \end{align}
 其中~$p(x,y)$~表示两个变量~$X$~和~$Y$~的联合概率密度函数。
~$X$~是基因表达量数据,其中的元素表示不同条件~(样本)中相应基因的表达值。
~$p(x)$~(或者~$p(y)$~)表示~$X$~~(或者~$Y$~)的边缘概率密度分布。

条件互信息~(CMI) 可以用熵表示为:
\begin{equation}
\begin{split}
CMI(X,Y|Z) &= H(X,Z) + H(Y,Z)\\
               & - H(Z) - H(X,Y,Z)
\end{split}
\end{equation}
其中 $H(X,Z)$, $H(Y,Z)$, $H(X,Y,Z)$ 表示联合熵。
CMI~值越高,表明给定变量~$Z$~,变量~$X$~和~$Y$~之间越可能存在密切关系。

熵可以用高斯核概率密度来估计~\cite{basso2005reverse},变量~$X$~的熵可以通过如下方式计算, 
其中 $|C|$ 是变量 $X$ 协方差矩阵的行列式~\cite{zhang2011inferring}:
\begin{equation}
    H(X) = log(2\pi e )^\frac{n}{2} |C| ^ {-\frac{1}{2}}
\end{equation}

进一步地,我们可以得到下面的等式:
\begin{equation}
    MI(X,Y)=\frac{1}{2}log\frac{|C(X)|*|C(Y)|}{|C(X,Y)|}
\end{equation}

\subsubsection{Loc-PCA-CMI}
众所周知,生物系统中节点之间是很少完全连通的,
大多数节点只直接连接到少量其它节点~\cite{jeong2000large},因此~GRN~也是一种稀疏网络。
识别网络的稀疏结构的关键步骤是识别可能具有相对高的共表达值的有意义的边。
具体来说,我们提出的方法~Loc-PCA-CMI~首先通过~Pearson~相关分析和~$p$~值错误率~(FDR)校正选择头部的~$n$~条高度共表达的边;
然后, 在缩减的边构成的空间中, 用边连接的基因计算局部重叠簇。
然后对于每个局部类, 我们应用~PCA-CMI~算法,
它可以通过从低到高依赖关联重复去除最可能不相关的边来构建高置信度无向网络~\cite{spirtes2000causation},
直到没有边可以删除,获取每个局部的网络结构。
通过对每个推断的局部子网络结构边权重取平均来获得完整调控网络的最终边的权重。
整个方法框架如图~\ref{pca-cmi-fr}~所示,实现细节如算法~\ref{alg}~所示。
由于~PCA-CMI~非常适合相对较小的~GRN~结构推断,我们进行了预处理:如果局部类中基因的数量小于或等于常数~$c$,
则直接应用~PCA-CMI~推断~GRN~的结构。
\begin{figure}[!htbp]
    \centering
    \includegraphics[width=0.95\textwidth]{pca-cmi-framework.pdf}
    \caption{Loc-PCA-CMI~方法框架图.
     a) 从基因表达矩阵~$M$~中抽取头部~$n$~条共表达边,其对应的候选基因为~$g_1,g_2,\ldots,g_{p}~$。
    这些候选基因被分组到局部簇中~$LOC_{M_1},~LOC_{M_2},\ldots,LOC_{M_{p}}$,
    其中~$g_1,g_2,\ldots,g_{p}$~是分别作为每个簇的质心。
    b) 对每一个局部之间有重叠的簇,我们应用~PCA-CMI~算法来得到它准确的结构。
    c) 聚合~$G_1$,~$G_2$,~\ldots,~$G_p$~来得到~GRN~的最终结构图~$G$。
    }
    \label{pca-cmi-fr}
\end{figure}

\begin{algorithm}[!htbp]
    \caption{Loc-PCA-CMI} %?????
    \label{alg}
    {\bf Input:} %?????? \hspace*{0.02in}??????????? \\ ????
    $M$ (the gene expression data matrix), $m$ (the number of genes), $n$ (the number of top ranked edges), $c$ (constant number); $k$ (CMI order number) and $\beta$ (order threshold) in subroutine PCA-CMI.
    
    {\bf Output:} %???????
    Graph weight matrix $G$ 
    %\begin{spacing}{1.5}
    \begin{algorithmic}[1]
    \If {$m \leq c$} 
    \State $G$ $\leftarrow$ PCA-CMI$(M, k, \beta)$;
    \State \Return $G$
    \Else
    \State Construct pair-wise gene-gene Pearson correlation matrix $\Omega = \rho(M_i, M_j)$;
    \State Select top $n$ edges as $E$ with highest Pearson correlation value in $\Omega$ with FDR correction in p-value, and according to which to get 
    $p$ candidate genes as $g_1,g_2,...,g_{p}$;
    \For{each gene in $g_1,g_2,...,g_{p}$} 
      \State Retrieve its directly connected genes that in edges list $E$ as local cluster $LOC_{M_i}$;
    \EndFor
    \For{each cluster $LOC_{M_i}$  in $LOC$}
      \State $G_{i}$ $\leftarrow$ PCA-CMI$(LOC_{M_i}, k, \beta)$;
    \EndFor
    \State $G$ $\leftarrow$ mean$(G_{1},G_{2},...,G_{p})$;
    \State \Return $G$
    \EndIf
    \end{algorithmic}
    %\end{spacing}
\end{algorithm}

算法~\ref{alg}~的计算复杂度通常由两个因素决定:~第一个是局部重叠簇~$p$~的数量,通常低于基因数~$m$; 第二个是~PCA-CMI~子程序的复杂性。
所有~PCA-CMI~子程序可以并行执行, 以使所提出的算法~Loc-PCA-CMI~更有效地执行。
$t$~与基因数量~$m$~的顺序相同。
~PCA-CMI~的计算复杂度由~CMI~阶数~$k$~和~$LOC_{M_i}$~的簇大小~$|C|$~控制,可以粗略估计为~$O(|C|^k)$~。
因此,PCA-CMI~的最终计算复杂度可以计算为~$O(m *|C|^k)$。
在最坏的情况下, 如果簇大小~$|C|$等于$m$, 即每个簇包含其中的所有基因, 因此计算复杂度为~$m*m^k = m^{k+1}$。 
然而, 这种最糟糕的情况在实验中很少发生, 因为~$|C|$通常低于~$m$。

\subsection{数据集}

我们使用来自著名的~DREAM3~竞赛的六个模拟数据~\cite{schaffter2011genenetweaver}对~Loc-PCA-CMI~的性能进行了基准测试。
~DREAM3~使用~GeneNetWeaver~软件来模拟生成基因网络表达数据。
在来自已知模式生物的调节相互作用系统的子网:~Ecoli~和~Yeast~的基础上,得到测试使用的基准网络。 
在我们的实验中评估了~DREAM3~中的总计六个基因敲除表达网络,
其包括三种不同规模的网络节点数:~10,~50,~100, 和两种不同类型的模式生物:~Ecoli~和~Yeast。

表~\ref{tbl}~展示了这~6~个数据集的详细描述。
\begin{table} [!htbp]
\caption{Descriptions of the datasets in our experiments} 
\label{tbl} 
\begin{center}
\resizebox{\columnwidth}{!}{%
\begin{tabular} {cccccc} 
\toprule
Datasets  & Number of samples & Average(Max) degree & Number of edges & Network density \\ 
\midrule
DREAM3-10 Ecoli &11 & 2.2(5) & 11 & 0.244\\
DREAM3-50 Ecoli & 51 & 2.48(14) & 62 & 0.051\\
DREAM3-100 Ecoli & 101 & 2.5(14) & 125 & 0.025\\
DREAM3-10 Yeast &11 & 2(4) & 10 & 0.222\\
DREAM3-50 Yeast & 51 & 3.08(13) & 77 & 0.063\\
DREAM3-100 Yeast & 101 & 3.32(10) & 166 & 0.034\\
\bottomrule
\end{tabular}
}
\end{center}
\end{table} 

针对每个数据集而言,输入数据文件中的行代表样本~(实验),~列代表基因~(实验变量)。
第一行是野生型表达数据,该样本中的每个基因都保持稳定状态。
第~$l$~($l>1$)~行则表示在对应的样本中第~$l-1$~个基因敲除后其它基因的表达量。

\subsection{结果和讨论}

如算法~\ref{alg}~中所述,三个参数影响了~GRN~结构推理中~Loc-PCA-CMI~的性能。
第一个参数是顶部选定边的数~$n$。
如果~$n$增加,则考虑更多边,随后局部簇大小将增加。
第二个参数是~$\beta$,它作为~MI~和~CMI~的阈值来决定独立性。
第三个参数是~CMI~阶号$k$。
从理论上讲,通过增加~$k$,如果~CMI~没有达到~$k-1$~阶的门槛~$\beta$,结构会更准确。
后两个参数是~PCA-CMI~和~PCA-PMI~里面的参数。
$n$的最佳值可以通过交叉验证获得,通常~$n$~的较大值可以促成更大规模的群集,并且网络中涵盖更多基因;
在我们的实验中,我们将其值均匀地设置为~$n =\binom{m} {2}/5$。
除了上述三个参数外,我们在算法~\ref{alg}中设置常量~$c$ = 10,
即如果基因数小于或等于~10,则~Loc-PCA-CMI~直接称为~PCA-CMI,
并且在这种情况下,Loc-PCA-CMI~和~PCA-CMI~的性能是相同的。
我们通过评估接收器工作特性曲线下面积~(AUROC)~和准确率召回率曲线下面积~(AUPR)~来评估~Loc-PCA-CMI~的性能。
与稀疏生物网络一样,不存在的边~(负样本)的数量明显超过现有边~(正样本)的数量; 
事实上,~AUPR~对~AUROC~\cite{saito2015precision}提供了更多信息。
我们倾向于使用~AUPR~进行评估,但为了与采用~AUROC~作为评估指标的其它方法进行保守比较,
我们还将~AUROC~作为补充指标。
较高的~AUROC~和~AUPR~值表明更准确的~GRN~预测。
为此,我们通过比较黄金标准网络中的监管边与最高~$q$~边来计算真阳性~(TP)、真阴性~(TN)、假阳性~(FP)和假阴性~(FN)边的数量来自~Loc-PCA-CMI~的排名列表输出。
通过绘制真实阳性率~TPR = TP/(TP + FN)~与假阳性率~FPR = FP/(FP + TN)~来增加~$q$~($q = 1, 2, \ldots, m^2$)来构建~ROC~曲线。
类似地,绘制精度~(TP/(TP+FP))~和召回~(TP/TP+FN)曲线以增加$q$。
应该注意的是,在算法~\ref{alg}~中,在获得每个局部聚类之后, PCA-CMI~和~PCA-PMI~都是后续结构粹化的候选方案。
如果用~PCA-PMI~替换~PCA-CMI,则会生成一种新方法,我们将其命名为~Loc-PCA-PMI,类似地。
然后得到四种基于~PCA~的方法,包括~PCA-PMI、PCA-CMI、Loc-PCA-PMI和~Loc-PCA-CMI,
所有这些方法目前都属于无模型方法。
如表~\ref{tbl}~所示,在六个基准数据集~DREAM3-10~中,~Ecoli~和~Yeast~数据集仅包含~10~个基因,
因此根据算法~\ref{alg}~的原理,
Loc-PCA-CMI~和~PCA-CMI~的表现相同,就像~Loc-PCA-PMI~和~PCA-PMI~的情况一样。
为了对这些基于~PCA~的方法进行有意义的比较,
我们选择了其它四个基因数大于~10~的数据集。
阶数在~\cite{zhang2011inferring,zhao2016part}~中有所讨论,
我们在实验中设置~$\beta$ = 0.03,~$k$ = 2。
同时为了探讨阶数~$k$~的大小是如何影响这些方法的性能,
在固定阈值~$\beta$ = 0.03~后,我们把这四种方法中的阶数~$k$~从~1~逐渐变化为~10,
然后分别计算~AUROC~和~AUPR。
图~\ref{fig:k}~总结了基准数据集的结果, 我们可以从此图得到以下两个结论:
\begin{itemize}
    \item 阶数~$k$~会对这四种基于~PCA~的方法的结果产生影响,通常当~$k$~达到~4~时~AUPR~和~AUROC~变得稳定,除了DREAM3-100 Ecoli数据集上略有不同。
    \item ~Loc-PCA-CMI~和~Loc-PCA-PMI~分别比~PCA-CMI~和~PCA-PMI~产生更高的~AUPR~和~AUROC,
    因此该算法采用的局部聚类策略有助于提高方法~PCA-CMI~和~PCA-PMI~的表现。
\end{itemize}

  \begin{figure*}[!htbp]
    \centering
    \begin{minipage}[b]{0.45\linewidth}
      \centering
      \centerline{
        \includegraphics[width = \linewidth]{K_Dream50_Ecoli.png}}
      \centerline{(A) DREAM3-50 Ecoli}
      \medskip  
    \end{minipage}
    \begin{minipage}[b]{0.45\linewidth}
      \centering
      \centerline{
        \includegraphics[width =\linewidth]{K_Dream100_Ecoli.png}}
      \centerline{(B) DREAM3-100 Ecoli}
      \medskip  
    \end{minipage}
      \begin{minipage}[b]{0.45\linewidth}
      \centering
      \centerline{
        \includegraphics[width = \linewidth]{K_Dream50_Yeast.png}}
      \centerline{(C) DREAM3-50 Yeast}
      \medskip  
    \end{minipage}
    \begin{minipage}[b]{0.45\linewidth}
      \centering
      \centerline{
        \includegraphics[width =\linewidth]{K_Dream100_Yeast.png}}
      \centerline{(D) DREAM3-100 Yeast}
      \medskip  
    \end{minipage}
    \caption{AUPR and AUROC  by varying $k$ from 1 to 10 of four PCA based methods on four different datasets: 
    (A) DREAM3-50 Ecoli; 
    (B) DREAM3-100 Ecoli; 
    (C) DREAM3-50 Yeast; 
    (D) DREAM3-100 Yeast.}
    \label{fig:k}
    \vspace{-0.5em}
  \end{figure*}

我们在六个基准数据集上使用~Loc-PCA-CMI~和四种基准方法~ARACNE、MRNET、PCA-PMI、PCA-CMI~进行了比较实验。
我们使用~R~包~``minet" 和其默认参数来评估~ARACNE~和~MRNET~\cite{meyer2008minet}。
使用~Pearson~相关系数直接从连续基因敲除表达数据~\cite{olsen2008impact,meyer2010information}~中近似估计方法的~MI~矩阵。
为了实现~PCA-PMI~和~PCA-CMI, 我们根据~\cite{zhang2011inferring,zhao2016part}~中提供的~URL~下载了~MATLAB~代码。
另外,PCA-PMI~和~PCA-CMI~方法中的参数使用默认值,也就是~$\beta$ = 0.03~和~$k$ = 2。
对于~Loc-PCA-CMI, 我们还对这两个参数采用了相同的值进行比较。
表~\ref{tab:performance_comparison}~给出了该实验的~AUROC~和~AUPR。
从表中可以看出,当网络规模增大时, 所有的方法的~AUPR~都会急剧下降。
Loc-PCA-CMI~仅在~DREAM3-10~Yeast~数据集中的~PCA-PMI~(或本文提出的~Loc-PCA-PMI~)之后,
而在其它五个数据集中,就~AUROC~和~AUPR~而言,
Loc-PCA-PMI~表现优于其它四种方法~ARACNE、MRNET、PCA-PMI~和~PCA-CMI。
此外,为了更完整地比较,我们还在表中展示了~Loc-PCA-PMI~的实验结果,
其中~$\beta$ = 0.03~和~$k$ = 2。
Loc-PCA-CMI~和~Loc-PCA-PMI~在~AUROC~上几乎相同。
然而,在大多数数据集中,~Loc-PCA-CMI~的~AUPR~优于~Loc-PCA-PMI。
我们提供了包括所有方法、基准数据集和测评脚本相关的资料, 
在~\url{https://github.com/chenxofhit/Loc-PCA-CMI.git}~上可开放获取。

\begin{table}[!htbp]
    % \scalebox{0.9}{
    % \begin{minipage}{1.1\linewidth}
    \resizebox{\columnwidth}{!}{%
      \centering  
      \begin{threeparttable}  
      \caption{AUROC and AUPR for the six datasets using different methods}  
      \label{tab:performance_comparison} 
        \begin{tabular}{ccccccccccccc}  
        \toprule  
        \multirow{2}{*}{Dataset}&  
        \multicolumn{2}{c}{ARACNE}&\multicolumn{2}{c}{MRNET}&\multicolumn{2}{c}{PCA-PMI}&\multicolumn{2}{c}{Loc-PCA-PMI}&\multicolumn{2}{c}{PCA-CMI}&\multicolumn{2}{c}{Loc-PCA-CMI}\\
        \cmidrule(lr){2-3} \cmidrule(lr){4-5}  \cmidrule(lr){6-7}  \cmidrule(lr){8-9}  \cmidrule(lr){10-11}  \cmidrule(lr){12-13} 
        &AUROC&AUPR &AUROC&AUPR &AUROC&AUPR &AUROC&AUPR &AUROC&AUPR &AUROC&AUPR\\
        \midrule  
        DREAM3-10 Ecoli  & 0.523 &0.255   &0.518&0.258    &0.816&0.483    &0.816&0.483    &{0.825}&{0.499}   &\textbf{0.825}&\textbf{0.499}\\
        DREAM3-50 Ecoli  &0.474 &0.050    &0.529&0.061    &0.828&0.385    &\textbf{0.846}&0.393    &0.825&0.396 &0.845&\textbf{0.422}\\
        DREAM3-100 Ecoli &0.505&0.027     &0.488&0.025    &0.857&0.299    &0.865&0.301    &0.851&0.311       &\textbf{0.865}&\textbf{0.336}\\
    
        DREAM3-10 Yeast  &0.628&0.321     &0.644&0.322    &0.995&0.933    &\textbf{0.995}&\textbf{0.933} &0.993&0.918 &0.993&0.918\\
        DREAM3-50 Yeast  &0.507&0.074     &0.524&0.080    &0.844&0.408    &0.869&0.417 &0.820&0.406   &\textbf{0.871}&\textbf{0.444}\\
        DREAM3-100 Yeast &0.547&0.040     &0.556&0.042    &0.863&0.368    &\textbf{0.871}&0.374 &0.854&0.389   &0.870&\textbf{0.409}\\
        \bottomrule  
        \end{tabular}  
        \end{threeparttable}  
        %   \end{minipage}
        %   }
        }
    \end{table} 

\subsection{结论}
我们基于分而治之的想法提出了一种命名为~Loc-PCA-CMI~的新的基于无模型的~GRN~结构推断方法。
在~DREAM 3~敲除数据集上的实验表明,~Loc-PCA-CMI~受益于局部聚类的策略。
此外,~Loc-PCA-CMI~优于其它方法,
包括~ARACNE、MRNET、PCA-PMI~和~PCA-CMI, 特别是在大小为~50~和~100~的网络上表现更佳。

Loc-PCA-CMI~是~PCA-CMI~的扩展版本,其计算效率的局限性也一样存在,
特别是在处理大型数据集时。
在大型网络的情况下, 局部簇的数量可能非常大。
但是,如果我们可以控制每个局部簇的大小,我们的方法也将适用于大型数据集。
我们未来的工作之一是改进聚类策略,
例如整合蛋白质复合物~\cite{li2017identification, li2017dynetviewer},
以便更有效地处理大规模样本数据。
另外,值得注意的是,我们主要关注推断~GRN~的结构,并没有考虑网络自身的稳定性问题。
因此,我们未来的研究将尝试从网络稳定性的角度出发推断更鲁棒的基因调控网络结构。