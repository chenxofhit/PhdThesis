\section{总结与展望}

\subsection{研究工作总结}

作为生物分化过程和生长机制中的重要一环,
基因调控网络的研究是系统生物学的基础和热点问题。
基因调控网络是通过测量基因的表达值进行预处理,
按照设计的推断策略,并结合已知的先验知识,构建得出基因之间相互作用的网络。
技术上,近十几年来,工业界和学术界在大数据、机器学习等技术上的积累和应用也十分成熟,
这为我们解决系统生物学最核心基础的基因调控网络构建的问题提供了强大的技术支撑。
数据上,这些年随着测序技术的快速发展,
基因表达数据的获取成本大幅降低,生成速度大幅加快,数据来源越来越多,数据种类越来越丰富, 数据规模也越来越大。
而且,单细胞技术的出现和成熟,为研究者从单个细胞层次测定基因的表达量提供了可能,减少了传统技术的噪声,
为基因调控网络的构建提供了更加细粒度精确的数据。
单细胞技术还能够获得不同类型和周期细胞的表达数据,
这为研究特定细胞类型中的调控网络以及参与细胞分化过程的调控网络提供了可能。
在技术和数据的双重加持下,
研究者可以利用基因调控网络来探究基因间的调控关系,发掘隐藏的功能特征,最终应用于生物信息、医学制药等领域。
目前,一个不容忽视的问题是,基因调控网络的预测准确性还有很大的提高空间,
许多疾病的发病机理以及细胞间的相互作用还有待发掘。
随着技术的进步,基因调控网络的研究经过结合计算机、医学、生物学和数学等科学,形成交叉学科,
基因调控网络在今后的生物研究中仍将会获得持续的关注, 扮演着极其重要的角色。

本文在总结和分析已有的基因调控网络构建的基础上,主要的工作如下:

(1) 基于信息理论的各种~GRN~推断方法被提出来。
然而,由于原始数据中存在的外部噪声、
网络结构中的拓扑稀疏性和非线性基因之间的依赖等因素,
这些方法在网络推断中会引入冗余的依赖关系。
特别是随着网络规模的增加,这些方法的表现大幅降低。
我们提出了一种新的网络结构推断方法~Loc-PCA-CMI: 首先识别局部重叠基因簇,
然后基于条件互信息~(PCA-CMI)的路径一致性算法推断每个簇的局部网络结构,
最终通过聚合局部网络结构,也就是基因之间的依赖性网络,来构造最终的~GRN。
我们在~DREAM3~敲除数据集上对~LOC-PCA-CMI~进行了评估,
将其性能与其它基于信息理论的网络结构推断方法,包括~ARACNE、MRNET、PCA-CMI和PCA-PMI,进行了比较。
实验结果证明,Loc-PCA-CMI~在~DREAM3~数据集上特别是在基因数目为~50~和~100~的网络上表现优于其它四种基准方法。

(2) 基于回归的各种~GRN~推断方法被提出来。
这几年的关于数据驱动的动态网络构建的研究,为我们提供了解决回归问题的新视角。
我们提出了一种数据驱动的动态网络构建方法来推断基因调控网络~(D3GRN)。
其中,将每个目标基因的调控关系转化为功能分解问题,并利用揭示网络相互作用的算法~(ARNI)~解决各个子问题。
为了弥补~ARNI~仅从单元级构建网络的局限性,我们采用抽样和基于面积的评分方法来推断最终的网络。
在~DREAM4和~DREAM5~基准数据集上, D3GRN在~AUPR~这个评价指标上与最先进的算法相比具有竞争力。

(3) 构建单细胞调控网络之前一个具有挑战性的问题是如何从超大规模的~scRNA-seq~数据中识别稀有细胞和区分细胞类型。
现有的寻找稀有细胞的算法非常耗时或耗费内存。
我们提出了一种高效准确的方法~DoRC~(\underline{D}iscovery \underline{o}f \underline{R}are \underline{C}ells)。
DoRC~产生的稀有度分数可以帮助生物学家们着重于下游分析,只对超大规模内的部分表达细胞~scRNA-seq~数据进行分析。
为了在随后的下游分析的过程中区分细胞类型,我们提出了新颖有效的细胞聚类方法~RafClust。
我们使用~${\sim}68$k~人血细胞的单细胞表达谱还证明了~DoRC~在划分人类血液树突状细胞亚型方面的效果。
DoRC~可以识别仿真数据集里面的稀有细胞,并且对细胞类型特征也很敏感。

(4) 构建单细胞基因调控网络中, 
识别细胞类型特征和细胞的基因表达活动程序~(如生命周期过程、对环境因素的反应)~对于理解细胞和组织的组成至关重要。
虽然单细胞~RNA-Seq~(scRNA-Seq)~可以量化成个体细胞中的转录本,
每个细胞的表达谱可能是这两种类型的程序的混合物,使它们难以分离。
我们提出了一个使用矩阵分解的算法~WSSMFA~来解决这个问题。
通过在公开的大脑类器官~scRNA-Seq~数据集上的实验表明,我们提出的~scGRNHunter~方法可以准确地推断出身份和活动性的子程序~(包括它们在每个细胞的相对贡献), 
并在此基础上构建成基于细胞类别身份的基因调控网络和基于细胞活动的基因调控网络。

\subsection{研究展望}
% 虽然相比之前的方法我们的预测结果有很大提高,在多个数据集上也表现良好,在网络结构未知的情况下对网络进行预测。
% 而网络本身是具有特定的结构和特性的,对于参数,数据集和网络结构比较敏感。
% 例如互信息本身对于某些网络的预测准确度非常高。
% 各类方法对于数据和参数非常敏感,可能某些方法对于特定数据集预测准确率非常高,
% 但这也许是一种过拟合现象,同时也说明算法本身对于该数据非常适用,
% 找出不用数据集中的网络由哪些结构组成,数据集本身具有哪些特性,
% 每种算法中的参数对于模型有多大的影响,就需要详细的实验分析。

(1) 基于互信息的调控网络结构推断改进

互信息是基于变量之间的依赖程度,从数值上来看是概率的大小,
而从结构上来看就是变量之间的位置关系和紧密程度,MI~只考虑两两之间的信息量,
而~CMI~则考虑三者之间的关系,运用联合概率提高了预测精确度。
因此,一方面可以继续分析变量结构关系尝试改进,去识别网络中的间接调控作用。
另一方面从基因的网络结构入手,例如某些网络的部分结构很多方法都无法预测正确,
那么是否可以分析该结构来设计适用的算法,比如针对常见的~FFT motif~结构, 
然后针对性地和现有算法结合来提高准确度。

(2) 基于回归的调控网络推断改进

回归方法除了能推断出基因调控网络的结构之外,更吸引人的是它能够推断出基因之间的相互作用的方向~(即上调和下调)。
回归模型将基因调控建模转化为机器学习特征选择的问题,
即是将靶标基因的表达看作是调控基因表达之间的相互线性作用或者非线性作用的结果。
它们应用在基因调控网络构建上优点是计算效率高,网络构建准确率高,
缺点是一些非线性的模型可解释性较低参数意义不明确,缺少对生物结构的支持。
D3GRN~尝试着从数据驱动的角度构建基因调控网络, 注重可解释性,但是计算效率上只适合小型网络。
针对大型的基因调控网络推断,还需要借助于深度学习等技术来构建回归模型,但是需要注重深度模型的可解释性。

(3) 基于深度学习的单细胞填充和聚类方法改进

单细胞数据填充和聚类是单细胞数据上游分析的核心任务, 
在很大程度上是构建与细胞类型相关的基因调控网络的前置条件。
随着高通量测序技术的发展, 需要处理的细胞数以万计, 同时大规模的单细胞表达数据也极度稀疏,
由于深度学习方法先天具有计算上的高效率,
在填充上, 现在的主流方法大部分是基于矩阵分解和统计模型的,
如果将矩阵分解和深度学习计算模型相结合,同时考虑融合先验知识,
比如基因和基因之间的相互作用~(比如调控,或者是共表达),
将会给单细胞的数据填充带来革命性的变化。
在聚类上,现有的方法比如~Seurat~和~SC3~在处理大规模单细胞数据集时稍显不足,
即使是本文提出的~RafClust~方法也仅仅适合中等规模的数据集。
基于图的聚类算法和基于深度表示学习的聚类算法,
结合比如细胞的位置信息~(Spatial information), 将会有极大的学术价值和产业应用前景。
 
(4) 