\section{总结与展望}

\subsection{研究工作总结}

作为生物分化过程和生长机制中的重要一环,
基因调控网络 (GRNs) 的研究是系统生物学的基础和热点问题。
基因调控网络构建是通过测量基因的表达值进行预处理,
按照设计的推断方法,并结合先验知识,构建得出基因之间相互作用的网络。
特别是,近二十年来工业界和学术界在大数据、机器学习等技术上的积累和应用十分成熟,
这为我们解决系统生物学最核心基础的基因调控网络构建的问题提供了强大的技术支撑;
另外,测序技术的发展十分迅速,
基因表达数据的获取成本大幅降低,生成速度大幅加快,数据来源越来越多,数据种类越来越丰富, 数据规模也越来越大。
而且,单细胞技术的出现和成熟,降低了传统技术中的噪声,使得研究者从单个细胞层次测量基因的表达量成为了现实,
进而为基因调控网络的构建提供了更加细粒度和精准的数据。
单细胞技术还能够获得不同类型和周期细胞的表达数据,
这为研究特定细胞类型中的调控网络以及参与细胞分化过程的调控网络提供了可能。
在技术和数据的双重加持下,
研究者可以利用基因调控网络来探究基因间的调控关系,发掘隐藏的功能特征,最终应用于生物信息、医学制药等领域。
目前,一个不容忽视的问题是,基因调控网络的预测准确性还有很大提高的空间,
许多疾病的发病机理以及细胞间的相互作用还有待发掘。
随着技术的进步,计算机、医学、生物学和数学等学科的交叉,
基因调控网络的构建在今后的生物研究中仍然会扮演着极其重要的角色,并会获得持续的关注。

本文在总结和分析已有的基因调控网络构建的基础上,主要的工作如下:

(1) 由于原始微阵列数据中存在的外部噪声、
网络结构中的拓扑稀疏性和非线性基因之间的依赖等因素,
这些方法在网络推断中会引入冗余的依赖关系。
特别是随着网络规模的增加,这些方法的表现大幅降低。
本文提出了一种基于互信息和局部结构的基因调控网络构建方法 Loc-PCA-CMI。
该方法根据基因的共表达关系来识别局部重叠基因簇,然后基于条件互信息 (PCA-CMI) 的路径一致性算法推断每个基因簇的局部网络结构,
最终通过聚合局部网络结构,也就是基因之间的依赖性网络,来构造最终的基因调控网络。
我们在 DREAM3 敲除数据集上对 Loc-PCA-CMI 进行了评估,
将其性能与其它四种基于信息理论的网络结构推断方法进行了比较。
实验结果表明, Loc-PCA-CMI 在 DREAM3 数据集上降低了构建中冗余的依赖关系,
特别是在基因数目为 50 和 100 的网络上在 AUPR 这个评价指标上表现优于其它四种方法。

(2) 当前基于数据驱动方法无法构建全局网络,
本文提出了一种基于数据驱动和抽样的基因调控网络构建方法 D3GRN。
% 该方法使用了抽样的策略 (bootstrapping),
% 将每个目标基因的调控关系转化为函数分解问题并利用揭示网络相互作用的算法解决各个子问题。
% 为了弥补数据驱动方法无法构建全局网络的缺陷,
% 我们最后采用了抽样策略和基于面积的评分方法来推断最终的网络。
该方法将针对每个目标基因的调控网络构建转化特征选择问题,
采用改进的数据驱动的方法 ARNI 来推断各个子网络。
该方法结合了抽样策略和基于面积的评分方法来聚合这些子网络从而构建最终的全局网络,
克服了传统的数据驱动方法无法构建全局网络的缺陷。
实验结果表明,在 DREAM4 和 DREAM5 基准数据集上,
 D3GRN 在 AUPR 这个评价指标上与其它方法相比具有竞争力。

(3) 当前在单细胞 RNA-seq 数据集上细胞聚类不准确,
本文提出了一种基于随机森林相似性学习的单细胞聚类方法 RafClust。
该方法使用多种相关性度量方法来刻画细胞的特征, 
然后使用随机森林回归模型进一步学习细胞与细胞之间的相似性矩阵,
基于相似性矩阵后采用层次聚类来决定细胞的最终类别。
实验结果表明,在十个单细胞数据集上, RafClust 在 ARI 上表现优于其它六种方法。

(4) 现有的寻找稀有细胞的算法大部分依赖单细胞聚类方法,
在处理超大规模 scRNA-seq 数据时候非常耗时或耗费内存,
本文提出了一种基于孤立森林的单细胞稀有细胞识别方法 DoRC。
该方法利用孤立森林高效地来对每个细胞产生稀有度分数,
结合阈值方法对细胞进行稀疏性的二元标注。
实验结果表明,在超大规模的单细胞 RNA-seq 数据 ${\sim}68$k 人血细胞的单细胞表达谱上,
DoRC 在划分人类血液树突状细胞亚型方面有突出的效果,执行效率高。
另外, DoRC 可以识别仿真数据集里面的稀有细胞,并且对细胞类型特征也很敏感。


(5) 在构建单细胞基因调控网络时,识别细胞类型特征和细胞的基因表达活动程序对于理解细胞和组织的组成至关重要。
当前从单细胞 RNA-seq 数据中无法同时推断出与细胞类型相关和与细胞活动相关的基因调控网络,
本文提出了一种基于矩阵分解的基因调控网络构建方法 scGRNHunter。
该方法利用我们提出的矩阵分解算法 WSSMFA 在单细胞 RNA-seq 数据上同时分离出细胞类型程序和细胞活动程序,
然后结合公开数据库 TRRUST 构建基于每个程序的基因调控网络。
实验结果表明,在公开的大脑类器官 scRNA-Seq 数据集上,我们提出的 scGRNHunter 方法可以有效推断出身份和活动性的子程序, 
并在此基础上构建基于细胞类型的基因调控网络和基于细胞活动的基因调控网络。


总结一下, 这五种方法涉及到 DNA 微阵列和单细胞 RNA-seq 这两种类型的基因测序技术数据。
针对传统的 DNA 微阵列数据, 本文提出了基因调控网络的结构推断方法 Loc-PCA-CMI 和基因调控网络的有向推断方法 D3GRN。
针对单细胞 RNA-seq 测序数据, 本文提出了面向单细胞测序数据的基因调控网络构建方法 scGRNHunter。
在不断逐步递进的研究过程中,本文发现由于单细胞测序数据稀疏, 基因调控网络构建更加困难,基因调控网络的构建与细胞类型有关。
所以,本文进一步研究了单细胞 RNA-seq 测序数据下的细胞类型识别问题: 首先提出了基于随机森林相似性学习的单细胞聚类方法 RafClust, 
在该方法的基础上针对稀有细胞识别这一更具有挑战性的问题,进一步地又提出了单细胞稀有细胞的识别方法 DoRC。

\subsection{研究展望}
% 虽然相比之前的方法我们的预测结果有很大提高,在多个数据集上也表现良好,在网络结构未知的情况下对网络进行预测。
% 而网络本身是具有特定的结构和特性的,对于参数,数据集和网络结构比较敏感。
% 例如互信息本身对于某些网络的预测准确度非常高。
% 各类方法对于数据和参数非常敏感,可能某些方法对于特定数据集预测准确率非常高,
% 但这也许是一种过拟合现象,同时也说明算法本身对于该数据非常适用,
% 找出不用数据集中的网络由哪些结构组成,数据集本身具有哪些特性,
% 每种算法中的参数对于模型有多大的影响,就需要详细的实验分析。

(1) 基于互信息的调控网络结构推断改进

互信息是基于变量之间的依赖程度,从数值上来看是概率的大小,
而从结构上来看就是变量之间的位置关系和紧密程度, MI 只考虑两两之间的信息量,
而 CMI 则考虑三者之间的关系,运用联合概率提高了预测精确度。
因此,一方面可以继续分析变量结构关系尝试改进,去识别网络中的间接调控作用。
另一方面从基因的网络结构入手,例如某些网络的部分结构很多方法都无法预测正确,
那么是否可以分析该结构来设计适用的算法,比如针对常见的 FFT motif 结构, 
然后针对性地和现有算法结合来提高准确度。

(2) 基于回归模型的调控网络推断改进

基于回归的方法除了能推断出基因调控网络的结构之外,更吸引人的是它能够推断出基因之间的相互作用的方向 (即上调和下调)。
回归模型将基因调控建模转化为机器学习特征选择的问题,
即是将靶标基因的表达看作是调控基因表达之间的相互线性作用或者非线性作用的结果。
它们应用在基因调控网络构建上优点是计算效率高,网络构建准确率高,
缺点是一些非线性的模型可解释性较低参数意义不明确,缺少对生物结构的支持。
如果是从数据驱动的动态网络构建角度构建基因调控网络, 注重可解释性,但是计算效率上只适合小型网络。
针对大型的基因调控网络推断,还需要借助于深度学习等技术来构建回归模型,但是需要注重深度模型的可解释性。

(3) 基于深度学习的单细胞填充和聚类方法改进

单细胞填充和聚类是单细胞数据上游分析的核心任务, 
在很大程度上是构建与细胞类型相关的基因调控网络的必要步骤。
随着高通量测序技术的发展, 需要处理的细胞数以万计, 同时大规模的单细胞表达数据也极度稀疏,
深度学习方法先天具有计算上的高效率,已经在各种生物学应用中取得成功,
包括基因组学、转录组学、蛋白质组学、结构生物学。
在填充上, 现在的主流的 dropout imputation 方法大部分是基于矩阵分解和统计模型的,
如果将矩阵分解和深度学习计算模型相结合,同时考虑融合先验知识,
比如基因和基因之间的相互作用 (比如调控,或者是共表达),
将会给单细胞的数据填充带来革命性的变化。
在聚类上,单细胞数据存在的批次效应对聚类的影响非常巨大,消除批次效应的影响,
使得同类型不同批次的细胞表达数据尽量对齐,
有研究者提出了使用深度学习而不是传统的统计学来消除单细胞测序中的批次效应的工具。
现有的聚类方法比如 Seurat Clustering \upcite{seuratv2} 和 SC3 \upcite{kiselev2017sc3} 在处理大规模单细胞数据集时稍显不足。
基于图的聚类算法和基于深度表示学习的聚类算法,
融合比如细胞的空间位置信息 (Spatial information)或者是多元单细胞组学数据比如
单细胞 ATAC-seq (single cell ATAC-seq, scATAC-seq)、单细胞 Hi-C (scHi-c), 
将会有极大的学术研究价值和产业应用前景。

(4) 基于深度学习的基因调控网络推断方法改进

虽然深度学习在生物信息学上取得了极大的应用成就,
将深度学习技术应用到基因调控网络还存在许多挑战,考虑到生物数据的可变性以及数据来源的不同,
在一个数据集上训练的模型可能无法很好地推广到其它数据集。
基于深度学习的方法需要大量的基因表达数据和已知的基因表达调控关系,
现在的数据集要么太小无法满足算法的要求,或者是像单细胞测序数据过于稀疏,
而且深度学习模型缺乏很好的可解释性,
导致深度学习模型在基因调控网络推断中未成为主流。
在下一阶段的研究中,可以从两个角度来着手考虑:
从数据角度看,可以考虑设计深度学习算法从图像数据中表征细胞的变化,
因为虽然单细胞基因表达数据稀疏但是图像数据却十分稠密,可以将两种异构数据结合起来;
或者考虑使用类似于 NLP 中的 Word Embedding (词嵌入模型) 对大规模稀疏的单细胞基因表达数据进行编码后进行后续处理;
或者是使用 VAE(变分编码器, \upcite{kingma2013auto}) 或者 GAN (生成对抗式网络, \upcite{goodfellow2014generative}) 等模型生成符合测序技术特点的模拟数据;
也或者是考虑基因表达数据融合 TF-gene调控数据集, PPI 数据, 和 DNA 甲基化数据等多模态多组学数据。
从计算角度看,可以把基因调控网络推断问题转化为深度学习模型擅长处理的分类问题 \upcite{Yuan27151}, 
已知的 TF 和 Gene 之间的调控关系当作正样本, 不存在的调控关系当作负样本,
TF 与 Gene 之间的调控关系预测转换为一个多分类的问题。